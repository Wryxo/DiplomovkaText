\chapter*{Úvod}
\addcontentsline{toc}{chapter}{Úvod}  
\markboth{}{Úvod}
\paragraph{}
V dnešnom svete fungujúcom na elektronických dátach, ktoré pre nás majú obrovskú cenu, sa ľudia snažia udržať ich čo najviac v tajnosti. Tieto dáta sa dajú jednoducho ochrániť, ak k ním bude mať prístup len majiteľ. Toto avšak nie je najpoužiteľnejšie riešenie keďže takmer každý počítač je pripojený do nejakej siete. Iná možnosť je mať uložené tieto dáta vo forme, ktorá bude dávať zmysel len povereným osobám, aj keď prístup k ním môžu mať aj iný ľudia. Na toto primárne slúži šifrovanie pomocou kľúča. 

\paragraph{}
Bezpečnosť tohto šifrovania záleží vysoko na utajený tohto kľúča. Preto je dôležité aby nebol ľahko odhadnuteľný. Keďže počítače priniesli so sebou obrovskú výpočtovú silu, sú schopné robiť až niekoľko desiatok tisíc pokusov za sekundu snažiac sa uhádnuť tento kľúč \cite{gpu25}. Keďže rýchlosť tohto hľadania kľúča záleží hlavne od veľkosti prehľadávaného priestoru kľúčov, v praxi sa bežne používajú aspoň 256 bitov dlhé kľúče. Toto je ekvivalent 32 znakového používateľského hesla zloženého zo ľubovolných znakov ASCII tabuľky.

\paragraph{}
Takéto hľadanie kľúča sa nazýva útok hrubou silou. Jeho podstatou je postupné generovanie možných kľúčov a následne overenie ich správnosti. V tejto práci sa budeme venovať skúmaniu a implementácií algoritmov na generovanie týchto hesiel. Program na následne overenie správnosti týchto hesiel nebudeme implementovať, keďže jeho implementácia by mala obsahovať množstvo optimalizácií, ktoré si mimo rozsah tejto práce. Výstup našej práce bude program, ktorý bude generovať zoznam hesiel, ktoré sa dajú následne použiť v niektorom z voľne dostupných programov na skúšanie takýchto hesiel ako napríklad \emph{hashCat}.