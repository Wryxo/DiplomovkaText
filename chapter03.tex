\chapter{Učenie}
Ako sme spomínali vyššie v texte v tejto práci sa budeme zaoberať útokom, ktorý dostane na vstupe slovník a na základe tohto slovníka bude generovať heslá zoradené podľa pravdepodobností. Kvalita výsledného zoznamu bude závisieť od schopnosti algoritmu správne sa naučiť ohodnotiť pravdepodobností jednotlivých reťazcov. V tejto práci kladieme dôraz na skúmanie možností generovania reťazcov použitím bezkontextových gramatík, avšak implementovali sme taktiež algoritmus používajúci Markovské zdroje, ktorý použijeme na porovnanie s bezkontextovými gramatikami.

\section{Pravdepodobnostné bezkontextové gramatiky}
\paragraph{}
Bezkontextové gramatiky sú definované štyrmi parametrami. Množinou neterminálov, ktoré slúžia ako premenné pri odvodzovaní vetnej formy. Množinou terminálov, ktoré tvoria reálny obsah výslednej vetnej formy. Túto množinu tvorí vstupná abeceda symbolov a je disjunktná s neterminálmi. Vetná forma obsahujúca len terminálne symboly sa nazýva terminálna vetná forma. Ďalej je potrebné zadefinovať počiatočný neterminál z ktorého sa bude každá vetná forma odvádzať. Nakoniec potrebujeme poznať množinu prepisovacích pravidiel, ktoré definujú spôsob akým sa menia neterminály na ďalšie vetné formy. Pri bezkontextových gramatikách majú prepisovacie pravidlá tvar
\[N -> (N \cup \Sigma)^*\]
kde N vyjadruje množinu neterminálov a \textSigma je množina terminálov. Tieto pravidlá vyjadrujú schopnosť neterminálu zmeniť sa na ľubovolnú vetnú formu, bez ohľadu na kontext v ktorom sa nachádza. V našej práci sa budeme venovať špeciálnym bezkontextovým gramatikám, ktorých každé prepisovacie pravidlo má priradenú pravdepodobnosť. Suma pravdepodobností jedného neterminálu bude vždy rovná 1. Vďaka týmto pravdepodobnostiam dokážeme ohodnotiť nami generované heslá a zoradiť ich podľa ich pravdepodobností. Pravdepodobnosť ľubovolnej vetnej formy získame súčinom pravdepodobností prepisovacích pravidiel použitých na jej odvodenie.

\paragraph{}
Odvodenia vetných foriem, čiže sekvencie použitých prepisovacích pravidiel, tvoria strom odvodenia danej gramatiky. Je možné aby v takomto strome existovali 2 rôzne cesty odvodenia, ktoré na koniec vygenerujú rovnakú vetnú formu. Tejto vlastnosti bezkontextových gramatík sa budeme snažiť vyhnúť vytvorením pravidiel tak aby ľubovolná terminálna vetná forma mala práve 1 spôsob odvodenia v danej gramatike. Zámerom tohto obmedzenia je zamedzenie generovania duplikátov, keďže predpokladáme, že ak pri skúšaní jedného hesla viac krát sa výsledok tohto pokusu nezmení.

\section{Markovské zdroje}
\paragraph{}
Bezkontextové gramatiky, ktoré sme implementovali si odvádzajú vetné formy výberom prepisovacieho pravidla s najvyššou pravdepodobnosťou. Taktiež si pamätajú, ktoré pravidlá už použili aby sa vyhli odvodeniu jednej terminálnej vetnej formy viac krát. Keďže gramatika si počas generovania reťazca pamätala celú postupnosť použitých prepisovacích pravidiel vyžadovala veľmi veľa pamäte. Preto sme implementovali náhodný proces prechádzajúci cez priestor stavov. Tento náhodný proces spĺňa Markovskú vlastnosť, ktorá je popísaná ako takzvaná bezpamäťovosť. Hovorí o tom, že distribúcia pravdepodobností následujúceho stavu závisí len od terajšieho stavu a nezáleží na sekvencií udalostí, ktoré mu predchádzali. Vďaka tejto vlastnosti je potrebná pamäť konštantná. Jediné čo si tento algoritmus pamätá, je tabuľka pravdepodobností pomocou ktorej sa rozhoduje aký najbližší symbol vygeneruje. Pre konštantne veľký prefix si Markovský zdroj poráta pravdepodobností následujúcich znakov. Pri tomto spôsobe učenia sa dá pre ľubovolný stav vypočítať pravdepodobnosť s akou sa Markovský zdroj dostane do tohto stavu.