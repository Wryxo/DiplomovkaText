\chapter{Implementácia}

\section{Tvorba gramatiky}
\paragraph{}
Pri tvorbe gramatiky sme potrebovali zaistiť aby gramatika spĺňala určité podmienky. Prvou z nich je schopnosť gramatiky vygenerovať všetky reťazce zo vstupnej abecedy kratšie ako používateľom zadaná maximálna dĺžka. Druhou podmienkou je aby každé terminálne slovo, ktoré gramatika generuje malo práve jeden strom odvodenia.

\paragraph{Jednodché neterminály}
Prvý typ neterminálov, ktoré budeme nazývať jednoduché, obsahuje pravidlá na zterminalnenie generovaného slova. Keďže naša vstupná abeceda obsahuje okolo 70 znakov, medzi ne patria veľke a malé písmena, cifry a niektoré často používané symboly, rozhodli sme sa ich rozdeliť do jednotlivých skupín. Pre každú z týchto skupín sme vytvorili neterminál, ktorý bude reprezentovať sekvenciu pevnej dĺžky zloženú zo znakov danej skupiny. V gramatike tieto neterminály vyjadrujeme pomocou prvého písmena anglického názvu danej skupiny.
\begin{itemize}
	\item U - upper case - veľké písmena
	\item L - lower case - malé písmena
	\item D - digit - cifry
	\item S - symbol - symboly
\end{itemize}
Každý jednoduchý neterminál sa teda skladá z písmena vyjadrujúceho skupinu znakov, ktoré generuje, a čísla popisujúceho dĺžku sekvencie na pravej strane pravidiel tohto neterminálu. Keďže generovanie všetkých variacií veľkostí k pri n prvkoch môže byť obrovské množstvo rozhodli sme sa zadefinovať maximálnu dĺžku sekvencie generovanej jednoduchým neterminálom.
\paragraph{Zložené neterminály}
Jednoduché neterminály nám pomáhajú vyjadrovať sekvenciu znakov práve jedného z vyššie vymenovaných typov. Aby sme boli schopný popísať ľubovolný reťazec tvorený znakmi vstupnej abecedy, budeme tieto jednoduché neterminály skladať do skupín, zložených neterminálov. Tieto neterminály vyjadrujú vzdy jeden možný predpis pre terminálne slovo. Napríklad neterminál \emph{U1L3D4} vyjadruje všetky terminálne slová začínajúce na veľké písmeno nasledované tromi malými písmena, ukončené štvoricou cifier.
\paragraph{}
Ďalej taktiež nedovoľujeme aby sa vyskytovali 2 jednoduché neterminály rovnakého typu za sebou. V prípade, že potrebujeme popísať sekvenciu terminálnych znakov jedného typu dlhšiu ako povolené maximum (popísane vyššie), rozdelíme túto sekvenciu do viacerých jednoduchých neterminálov pažravým algoritmom, čiže každý z týchto neterminálov zobere maximálný možný počet znakov sekvencie. Tento spôsob nám zaručí, že nevzniknú dva rôzne zložené neterminály vyjadrujúce ten istý predpis terminálneho slova.
\paragraph{}
Počiatočný neterminál gramatiky \emph{Z} bude obsahovať pravidlá prepisujúce tento neterminál na niektorý zo zložených alebo jednoduchých neterminálov. Tento spôsob generovania gramatiky spĺňa obe pravidlá, ktoré sme popisovali v úvode tejto kapitoly. 

\section{Počítanie pravdepodobností}
\paragraph{}
Ako sme spomínali v úvode textu, nami generované pokusy o nájdenie hesla chceme prispôsobit potrebám jednotlivých používateľom, ktorí sa snažia získať svoje stratené heslo. Aby sme vedeli čo najlepsie vyhovieť týmto používateľom, potrebujeme upraviť našu gramatiku. Tu prichádzajú do pozornosti pravdepodobností jednotlivých pravidiel našej gramatiky. Našim cieľom je nastaviť našu gramatiku tak aby generovala heslá podľa pravdepodobnosti použitia daným používateľom. Úspešnosť tohto učenia gramatiky bude drastický záležať od kvality vstupných dát.
\paragraph{}
Vzhľadom na to, že v dnešnom svete používatelia používajú rôzne služby, ktoré každá odporúča mať jedinečné heslo, používatelia používajú niekoľko hesiel naraz. Tieto heslá by si radi všetky pamätali a preto si často vytvoria pre seba charakteristický spôsob tvorby a zapamätania si týchto hesiel. V ideálnom prípade by sme chceli aby naše vstupné dáta pozostávali z čo najväčšieho počtu hesiel vytvorených pomocou tohto charakteristického spôsobu, keďže každé upresnenie informácií o hľadanom hesle nám zvýši rýchlosť nájdenia tohto hesla.
\paragraph{}
Keďže cieľom našej práce je nájsť heslo so 100\% pravdepodobnosťou, čo v najhoršom prípade znamená vygenerovať všetky možné reťazce kratšie ako zadaná maximálna dĺžka hesla, tak pravidla našej gramatiky sa budú meniť len pri zmene maximálnej dĺžky hesla. V ostatných prípadoch sa budú meniť len ich pravdepodobností. Pravdepodobností terminálných sekvencií budeme rátať ako percento výskytov danej terminálnej sekvencie spomedzi všetkých sekvencií spadajúcich pod tento neterminál. Práve kvôli tomuto spôsobu sme pridali v implementácií možnosť napísať do vstupného slovníku počty výskytov jednotlivých hesiel, aby mal používateľ možnosť zdôrazniť dôležitosť hesla. Pri počítaní pravdepodobností zložených neterminálov máme viacero možností ako postupovať.
\paragraph{Priamo zo vstupného slovníka}
Prvý spôsob ako postupovať bol identický s tým pre jednoduché neterminály. Pre každe pravidlo gramatiky prepisujúce neterminál Z na zvolený zložený neterminál vypočítame jeho pravdepodobnosť ako pomer počtu výskytov tohto neterminálu a všetkých výskytov. Tento spôsob môže mať ešte 2 varianty.
\begin{itemize}
	\item Do výskytov počítame len výskyty hesiel ktoré sú presne reprezentované daným neterminálom
	\item Do výskytov započítame aj výskyty kedy je zvolený neterminál podreťazcom iného neterminálu
\end{itemize}

\paragraph{Rekurzívne}
Ďalší spôsob spočíva v tom, že zo vstupného slovníka vypočítame pravdepodobností len pre jednoduché netermiály. Následne pre zložené neterminály počítame pravdepodobnosti ako súčin pravedpodobností jednoduchých neterminálov, ktoré daný neterminál obsahuje.

\paragraph{}
Všetky vyššie spomenuté metódy na počítanie pravdepodobností pravidiel gramatiky sme implementovali. Ich vzájomne porovnanie ako aj porovnanie s inými bežne používanými programami je vidieť v kapitole Výsledky.