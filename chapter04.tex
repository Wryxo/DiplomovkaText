\chapter{Implementácia}

\section{Tvorba gramatiky}
\paragraph{}
Pri tvorbe gramatiky sme sa držali niekoľkých pravidiel. Ako prvé sme potrebovali aby gramatika bola schopná vygenerovať všetky možné reťazce zo vstupného jazyka. Ďalej sme si dávali pozor, aby každé terminálne slovo bolo vygenerované práve raz, čiže aby každé terminálne slovo malo práve jeden strom odvodenia. Keďže naša vstupná abeceda obsahuje okolo 70 znakov – 52 veľkých a malých písmen, 10 cifier a približne 8 symbolov, rozhodli sme sa ich rozdeliť do jednotlivých skupín a pre jednotlivé skupiny vytvoriť neterminál, ktorý bude reprezentovať sekvenciu pevnej dĺžky zloženú zo znakov danej skupiny. Tieto neterminály sme si nazvali jednoduché. Teraz si zadefinujeme takzvané zložené neterminály, skladajúce sa zo série jednoduchých neterminálov. Tieto neterminály vyjadrujú jeden možný predpis pre terminálne slovo. Napríklad neterminál U1L3D4 vyjadruje všetky terminálne slová začínajúce na veľké písmeno nasledované tromi malými písmena, ukončené štvoricou cifier. Nakoniec vytvoríme počiatočný neterminál Z, ktorému pridáme pravidlá prepisujúce Z na niektorý zo zložených alebo jednoduchých neterminálov. Celá gramatika nakoniec vyzerá nasledovne: Z obsahuje pravidlá, ktoré vytvoria predpis pre požadované heslo a potom pravidlá, ktoré daný predpis prepíšu na terminálne slovo. Keďže každý možný reťazec znakov kratší ako maximálna dĺžka zapadá do práve jedného zloženého neterminálu, máme zaručené, že neexistujú dva rôzne stromy odvodenia pre jedno terminálne slovo. Zároveň existujú zložené neterminály pre všetky možné kombinácie jednoduchých neterminálov kratšie ako maximálna dĺžka reťazca, čiže budeme schopný vygenerovať všetky možné reťazce. Keďže gramatiku si budeme musieť niekam uložiť a určité jej časti budú musieť byť uložené aj v pamäti počítača, zadefinovali sme aj maximálnu veľkosť reťazcov vyjadrených v jednoduchých netermináloch. Ak chceme vygenerovať terminálne slovo obsahujúce sekvenciu znakov jedného typu dlhšiu ako povolené maximum, rozdelíme túto sekvenciu na jednoduchý neterminál maximálnej dĺžky a zvyšok sekvencie. Toto delenie opakujeme až dokým všetky jednoduché neterminály sú najviac maximálnej veľkosti. Tento spôsob delenia nám zaručí, že nevzniknú dva rôzne zložené neterminály vyjadrujúce ten istý predpis terminálneho slova.

\section{Počítanie pravdepodobností}
\paragraph{}
Ako sme spomínali v úvode textu, nami generované pokusy o nájdenie hesla chceme prispôsobit potrebám jednotlivých používateľom, ktorí sa snažia získať svoje stratené heslo. Aby sme vedeli čo najlepsie vyhovieť týmto používateľom, potrebujeme upraviť našu gramatiku. Tu prichádzajú do pozornosti pravdepodobností jednotlivých pravidiel našej gramatiky. Našim cieľom je nastaviť našu gramatiku tak aby generovala heslá podľa pravdepodobnosti použitia daným používateľom. Úspešnosť tohto učenia gramatiky bude drastický záležať od kvality vstupných dát.
\paragraph{}
Vzhľadom na to, že v dnešnom svete používatelia používajú rôzne služby, ktoré každá odporúča mať jedinečné heslo, používatelia používajú niekoľko hesiel naraz. Tieto heslá by si radi všetky pamätali a preto si často vytvoria pre seba charakteristický spôsob tvorby a zapamätania si týchto hesiel. V ideálnom prípade by sme chceli aby naše vstupné dáta pozostávali z čo najväčšieho počtu hesiel vytvorených pomocou tohto charakteristického spôsobu, keďže každé upresnenie informácií o hľadanom hesle nám zvýši rýchlosť nájdenia tohto hesla.
\paragraph{}
Keďže cieľom našej práce je nájsť heslo so 100\% pravdepodobnosťou, čo v najhoršom prípade znamená vygenerovať všetky možné reťazce kratšie ako zadaná maximálna dĺžka hesla, tak pravidla našej gramatiky sa budú meniť len pri zmene maximálnej dĺžky hesla. V ostatných prípadoch sa budú meniť len ich pravdepodobností. Pravdepodobností terminálných sekvencií budeme rátať ako percento výskytov danej terminálnej sekvencie spomedzi všetkých sekvencií spadajúcich pod tento neterminál. Práve kvôli tomuto spôsobu sme pridali v implementácií možnosť napísať do vstupného slovníku počty výskytov jednotlivých hesiel, aby mal používateľ možnosť zdôrazniť dôležitosť hesla. Pri počítaní pravdepodobností zložených neterminálov máme viacero možností ako postupovať.
\paragraph{Priamo zo vstupného slovníka}
Prvý spôsob ako postupovať bol identický s tým pre jednoduché neterminály. Pre každe pravidlo gramatiky prepisujúce neterminál Z na zvolený zložený neterminál vypočítame jeho pravdepodobnosť ako pomer počtu výskytov tohto neterminálu a všetkých výskytov. Tento spôsob môže mať ešte 2 varianty.
\begin{itemize}
	\item Do výskytov počítame len výskyty hesiel ktoré sú presne reprezentované daným neterminálom
	\item Do výskytov započítame aj výskyty kedy je zvolený neterminál podreťazcom iného neterminálu
\end{itemize}

\paragraph{Rekurzívne}
Ďalší spôsob spočíva v tom, že zo vstupného slovníka vypočítame pravdepodobností len pre jednoduché netermiály. Následne pre zložené neterminály počítame pravdepodobnosti ako súčin pravedpodobností jednoduchých neterminálov, ktoré daný neterminál obsahuje.

\paragraph{}
Všetky vyššie spomenuté metódy na počítanie pravdepodobností pravidiel gramatiky sme implementovali. Ich vzájomne porovnanie ako aj porovnanie s inými bežne používanými programami je vidieť v kapitole Výsledky.