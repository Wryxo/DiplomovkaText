\chapter{TrueCrypt}
\paragraph{}
TrueCrypt je šifrovací program poskytujúci používateľovi možnosť zašifrovať ľubovoľnú časť  disku v počítači pomocou používateľom zvoleného hesla. Vývoj tohto programu bol ukončený roku 2014 a podľa autorov nie je bezpečný, nakoľko jeho implementácia môže obsahovať bezpečnostné chyby. Cieľom tejto práce nie je odhalenie týchto chýb, nakoľko sa nesnažíme zlomiť toto šifrovanie. Našim cieľom je implementácia algoritmu schopného zistiť stratené používateľské heslo za účelom odšifrovania dát ich majiteľom. Aj napriek tomuto cieľu budeme naše pokusy o nájdenie heslá nazývať útoky na šifrovanie. V nasledujúcich kapitolách práce bližšie popíšeme algoritmus šifrovania disku pomocou programu TrueCrypt a následne podrobne rozoberieme možnosti útokov na používateľské heslá. V druhej polovici práci sa budeme venovať nami implementovanému algoritmu a jeho fungovaniu. Prácu ukončíme odsekom popisujúcim nami získane výsledky.
\paragraph{}
Ako sme spomínali v úvode program TrueCrypt slúži na zašifrovanie dát na používateľskom disku pomocou zvoleného hesla. Tento program má implementované viaceré šifrovacie algoritmy, avšak predpokladáme, že poznáme algoritmus zvolený na zašifrovanie cieľového disku a na základe toho ďalej v práci budeme pracovať s algoritmom AES-256. Zašifrovanie disku pomocou tohto programu prebieha v dvoch fázach. Najskôr vygeneruje náhodný kľúč a pomocou neho zašifruje disk aj s dátami. Následne zapíše konfiguráciu a vygenerované kľúče do hlavičky zašifrovaného disku a tú následne zašifruje kľúčom vygenerovaným pomocou používateľského hesla. Dôležitou informáciou o programe TrueCrypt je že zdrojové kódy tohto programu sú voľne dostupné na internete.
