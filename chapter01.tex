\chapter{TrueCrypt}
\paragraph{}
TrueCrypt je šifrovací program poskytujúci používateľovi možnosť zašifrovať ľubovoľnú časť disku v počítači pomocou používateľom zvoleného hesla. Vývoj tohto programu bol ukončený roku 2014 a podľa autorov nie je bezpečný, nakoľko jeho implementácia môže obsahovať bezpečnostné chyby. Cieľom tejto práce nie je odhalenie týchto chýb, nakoľko sa nesnažíme zlomiť toto šifrovanie. Našim cieľom je implementácia algoritmu schopného zistiť stratené používateľské heslo za účelom odšifrovania dát ich majiteľom. Nižšie bližšie popíšeme algoritmus šifrovania disku pomocou programu TrueCrypt a v následnovných kapitolách podrobne rozoberieme možnosti útokov na používateľské heslá.

\paragraph{}
Ako sme spomínali v úvode program TrueCrypt slúži na zašifrovanie dát na používateľskom disku pomocou zvoleného hesla. Tento program má implementované viaceré šifrovacie algoritmy, avšak predpokladáme, že poznáme algoritmus zvolený na zašifrovanie cieľového disku.  Zašifrovanie disku pomocou tohto programu prebieha v dvoch fázach. Najskôr vygeneruje náhodný kľúč a pomocou neho zašifruje disk aj s dátami. Následne zapíše konfiguráciu a vygenerované kľúče do hlavičky zašifrovaného disku a tú následne zašifruje kľúčom vygenerovaným pomocou používateľského hesla. Dôležitou informáciou o programe TrueCrypt je že zdrojové kódy tohto programu sú voľne dostupné na internete.