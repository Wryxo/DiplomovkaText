\chapter{Používateľské heslá}

\paragraph{}
Aj napriek tomu, ako sú používateľské heslá vo všeobecnosti ľahko prelomiteľné, sú dnes najčastejšie používaný spôsob autentifikácie používateľa. Tento trend sa pravdepodobne ani v najbližšej budúcnosti nebude meniť. Hlavným dôvodom slabých používateľských hesiel sú obmedzenia ľudskej pamäte. Ak by si používatelia nemuseli pamätať heslo, tak by používali heslá s najväčšou entropiu. Tie by boli najdlhšie možné povolené systémom, zložené z náhodne vybraných znakov povolených týmto systémom a neexistovala by žiadna iná možnosť ako túto sekvenciu dostať na základe inej informácie.

\paragraph{}
Tento spôsob tvorby hesiel je presný opak toho k čomu je prispôsobená ľudská myseľ. Ľudia sú schopní zapamätať si sekvenciu znakov dlhú približne sedem znakov plus mínus dva znaky vo svojej krátkodobej pamäti. Taktiež, aby si človek zapamätal takúto sekvenciu, táto sekvencia nemôže byť kompletne náhodná, ale musí sa skladať zo známych kusov informácie ako sú napríklad slová. Nakoniec ľudská myseľ funguje veľmi dobre vďaka redundancii informácie, čiže človeku sa ľahšie pamätajú veci, ktoré si vie odvodiť z viacerých iných kusov informácii.

\paragraph{}
Mnohé systémy používajúce heslá ako spôsob autentifikácie používateľa dávajú používateľovi rady ako si zvoliť bezpečné heslo. Na základe informácií spomenutých na začiatku tejto kapitoly by heslo malo byť dostatočne dlhé, skladajúce sa z rozumne veľkej abecedy znakov a malo by byť ľahko zapamätateľné. Väčšina týchto systémov sa sústredí hlavne na prvé dve podmienky tvorby hesla a to že by malo používateľove heslo spĺňať minimálnu dĺžku a obsahovať aspoň jeden z každej kategórie veľké, malé písmena, číslice a špeciálne znaky. Týmto dávajú dôraz na ochranu proti útokom hrubou silou oproti zapamätateľnosti hesla.

\paragraph{}
Mnoho používateľov, ktorí boli prezentovaný minimálnymi nárokmi na heslo, si vyvinulo spôsob na generovanie takýchto hesiel. Tento spôsob zahŕňal výber slova, ktorému zväčšili prvé písmeno a pridali sufix skladajúci sa z číslic a špeciálnych znakov. Takéto spôsoby zakladajúce na jednoduchej transformácii slova sa veľmi rýchlo ukázali neefektívne, keď sa počas posledného desaťročia náramne zvýšil výkon počítačov. Tie boli schopné skúsiť veľké množstvo transformácií pre každé slovo slovníka.

\paragraph{}
Začali sa objavovať mnohé mnemotechnické pomôcky umožňujúce generovať používateľské heslá. Jedna z často sa vyskytujúcich doporučovala vytvorenie si extrémne dlhého slovného spojenia. Účelom tejto metódy bola ochrana proti útokom hrubou silou zväčšením priestoru potenciálnych hesiel pomocou zvýšenia dĺžky samotného hesla. Ďalšia veľmi často používaná metóda bola založená na tvorbe hesla zobratím prvých znakov slov z fráze, ktorú používateľ vymyslel. Heslá založené na mnemotechnických pomôckach sa málokedy vyskytujú v slovníkoch používaných pri útokoch hrubou silou. To avšak neznamená, že sú bezpečnejšie ako bežné heslá \cite{Kuo:2006:HSM:1143120.1143129}.

\paragraph{}
V tejto práci sa snažíme vyvinúť algoritmus, ktorý dostane na vstupe slovník s heslami. Tento slovník ma slúžiť na naučenie algoritmu metódy tvorby a používania hesiel pre konkrétneho používateľa. Mal by zahŕňať ukážky hesiel, ktoré sú vytvorené podobnými metódami ako používateľ vytvára heslá pre potreby svojej autentifikácie.