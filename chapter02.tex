\chapter{Útoky hrubou silou}
\paragraph{}
Základným princípom útokov hrubou silou hľadanie správneho riešenia pomocou skúšania veľkého množstva kandidátov. Spôsob skúšania kandidátov sa môže líšiť od situácie, avšak veľmi často máme prístupnú zašifrovanú respektíve zahešovanú verziu hľadaného reťazca. Keďže hešovacie algoritmy sú dizajnované tak aby nebolo možné z hešu vyrobiť pôvodný reťazec, musíme pre vyskúšanie kandidáta zahešovať tohto kandidáta a následne porovnať výsledný heš s tým od správneho hesla. Metód akými sa dajú títo kandidáti generovať je mnoho a nižšie si predstavíme pár z nich.

\paragraph{}
Všetky v praxi používané algoritmy používajú kľúče dĺžky aspoň 256 bitov, čo je ekvivalent reťazca dĺžky 32 zloženého zo ľubovolných znakov ASCII tabuľky. V praxi je veľmi nepravdepodobné, že používateľ bude mať takéto dlhé heslo založené na tak veľkej abecede. Práve preto sa v tejto práci sústredíme na používateľské heslá, pretože majú omnoho menší počet možných reťazcov. Možností pre 256 bitový kľúč je 2\textsuperscript{256} zatiaľ čo možností pre 16 miestne heslo zložené z veľkých, malých písmen, číslic a niektorých často používaných znakov je približne 2\textsuperscript{101}, čo už je dosť signifikantné zmenšenie počtu možností.

\section{Inkrementálny útok}
\paragraph{}
Inkrementálna metóda patrí medzi najzákladnejšie útoky hrubou silou a často krát je práve to, čo sa myslí pod útokom hrubou silou. Podstatou tohto útoku je vyskúšanie všetkých kandidátov. Ak prejdeme cez celý priestor reťazcov, museli sme určite prejsť aj cez konkrétny reťazec, ktorý hľadáme. Táto metóda má tým pádom 100 percentnú úspešnosť. Jej problém avšak spočíva v množstve reťazcov, ktoré musíme vyskúšať. Vo väčšine prípadoch vieme obmedziť hľadanie maximálnou dĺžkou hľadaného výrazu a abecedou znakov z ktorej sa daný výraz skladá. Používateľské heslá mávajú maximálnu dĺžku okolo 16 znakov a sú zložene prevažne z asi 80 rôznych znakov. Pre takéto reťazce, ktorých je 80\textsuperscript{16}, by nám vygenerovanie všetkých trvalo približne 9\textsuperscript{16} rokov pri skúšaní 1 milióna reťazcov za sekundu.

\section{Slovníkový útok}
\paragraph{}
Častokrát existuje ešte menšie množina reťazcov, ktoré majú omnoho väčšiu šancu, že medzi nimi bude hľadaný výraz. Toto je pravda špeciálne pri hľadaní používateľských hesiel, nakoľko používatelia volia heslá aby boli zapamätateľné. Vďaka tomu existuje relatívne malá množina reťazcov, ktoré keď vyskúšame máme vysokú šancu úspechu. V takomto prípade je najlepšie zostaviť slovník takýchto reťazcov, ktoré potom postupne skúšame. Táto metóda väčšinou rýchlejšie nájde heslo ako vyššie spomínaná inkrementálna metóda. Avšak jej úspešnosť záleží hlavne od tohto vstupného slovníka. V dnešnom svete keď takmer každá služba vyžaduje heslo od používateľa, existuje veľa verejne prístupných zoznam najčastejšie používaných hesiel, ktoré slúžia ako veľmi dobrý základ pre tento útok.

\subsection{Prekrúcanie slov}
\paragraph{}
Samotný slovníkový útok väčšinou pokrýva takmer zanedbateľné percento všetkých možných výrazov spadajúcich do priestoru hesiel danej abecedy a dĺžky. Preto sa spolu s touto metódou často používa prekrúcanie slov. Podstatou je rozšírenie vstupného slovníka o alternatívne verzie vstupných hesiel za účelom rozšírenia prehľadaného priestoru reťazcov. Bežne sa to docieľuje definovaním zoznam pravidiel popisujúcich transformáciu slova. Tieto pravidlá budú následne aplikované na jednotlivé vstupné slová a tým vzniknú potenciálne nové reťazce, ktoré sa nenachádzajú vo vstupnom slovníku. Tieto pravidlá môžu transformovať slovo rôznymi spôsobmi od pridania prefixu či sufixu cez zmenu veľkostí písmen alebo vynechanie spoluhlások. Mnohé programy zaoberajúce sa útokmi hrubou silou podporujú vlastný jednoduchý jazyk na popísanie týchto pravidiel.

\section{Hybridný útok}
\paragraph{}
V tejto práci sa venujeme implementácií útoku, ktorý je spojením vyššie uvedených. Našim hlavným cieľom je nájdenie správneho hesla k particií zašifrovanej programom TrueCrypt. Keďže chceme toto heslo nájsť v ľubovoľné veľkom konečnom čase, budeme náš algoritmus implementovať, tak aby vygeneroval všetky možné reťazce zo vstupnej abecedy kratšie ako nami stanovená dĺžka. V tomto sa bude veľmi podobať na inkrementálny útok. Avšak v našom prípade predpokladáme, že na vstupe dostaneme ešte slovník obsahujúci zoznam hesiel. Predpokladáme, že tento zoznam je usporiadaný podľa pravdepodobnosti správnosti hesiel v ňom. Náš algoritmus si na základe týchto hesiel upraví pravdepodobností vygenerovania jednotlivých reťazcov aby následne mohol na výstup dávať heslá od najpravdepodobnejšieho. V práci sme implementovali 2 spôsoby ako spracovať vstupné dáta.