\chapter{Útoky hrubou silou}
\paragraph{}
Základným princípom útokov hrubou silou je hľadanie správneho riešenia pomocou skúšania veľkého množstva kandidátov. Spôsob skúšania kandidátov sa môže líšiť od situácie, avšak veľmi často máme prístupnú zašifrovanú, respektíve zahešovanú verziu hľadaného reťazca. Keďže hešovacie algoritmy sú dizajnované tak, aby nebolo možné z odtlačku vyrobiť pôvodný reťazec, musíme pre vyskúšanie kandidáta zahešovať tohto kandidáta a následne porovnať výsledný odtlačok s tým od správneho hesla. Metód akými sa dajú títo kandidáti generovať je mnoho a nižšie si predstavíme niektoré z nich.

\paragraph{}
Všetky v praxi používané algoritmy používajú kľúče dĺžky aspoň 256 bitov, čo je ekvivalent 32 znakového reťazca zloženého z ľubovolných znakov ASCII tabuľky. V praxi je veľmi nepravdepodobné, že používateľ bude mať takto dlhé heslo založené nad tak veľkou abecedou. Práve preto sa v tejto práci sústredíme na používateľské heslá, pretože majú omnoho menší počet možných reťazcov. Možností pre 256 bitový kľúč je 2\textsuperscript{256} zatiaľ čo možností pre 16 miestne heslo zložené z veľkých, malých písmen, číslic a niektorých často používaných znakov je približne 2\textsuperscript{101}, čo už je dosť signifikantné zmenšenie počtu možností (na \(2.18953*10^{-45} \%\) pôvodnej veľkosti).

\section{Útok úplným prehľadávaním}
\paragraph{}
Tento spôsob hľadania hesla patrí medzi najzákladnejšie útoky hrubou silou a častokrát sa práve on myslí pod pojmom útok hrubou silou. Podstatou tohto útoku je vyskúšanie všetkých kandidátov. Ak prejdeme cez celý priestor reťazcov, museli sme určite prejsť aj cez konkrétny reťazec, ktorý hľadáme. Táto metóda má tým pádom 100\% úspešnosť. Jej problém avšak spočíva v množstve reťazcov, ktoré je potrebné vyskúšať. Vo väčšine prípadov vieme obmedziť hľadanie maximálnou dĺžkou hľadaného výrazu a abecedou znakov z ktorej sa daný výraz skladá. Používateľské heslá mávajú maximálnu dĺžku okolo 16 znakov a sú zložené prevažne z asi 80 rôznych znakov. Pre takéto reťazce, ktorých je 80\textsuperscript{16}, by nám vyskúšanie všetkých trvalo približne \(8.92*10^{13}\) rokov pri skúšaní 1 miliardy reťazcov za sekundu.

\section{Slovníkový útok}
\paragraph{}
Častokrát existuje ešte menšia množina reťazcov, heslá z ktorej majú omnoho väčšiu šancu, že medzi nimi bude hľadaný výraz. Toto je pravda špeciálne pri hľadaní používateľských hesiel, nakoľko používatelia volia heslá tak, aby boli zapamätateľné. Vďaka tomu existuje relatívne malá množina reťazcov, ktoré keď vyskúšame, máme vysokú šancu úspechu. V takomto prípade je najlepšie zostaviť slovník takýchto reťazcov, ktoré potom postupne skúšame. Táto metóda väčšinou nájde heslo rýchlejšie ako vyššie spomínané úplne prehľadávanie. Avšak jej úspešnosť záleží hlavne od tohto vstupného slovníka. V dnešnom svete, keď takmer každá služba vyžaduje heslo od používateľa, existuje veľa verejne prístupných zoznamov najčastejšie používaných hesiel, ktoré slúžia ako veľmi dobrý základ pre tento útok.

\subsection{Prekrúcanie slov}
\paragraph{}
Samotný slovníkový útok väčšinou pokrýva takmer zanedbateľné percento všetkých možných výrazov spadajúcich do priestoru hesiel danej abecedy a dĺžky. Preto sa spolu s touto metódou často používa prekrúcanie slov. Podstatou je rozšírenie vstupného slovníka o alternatívne verzie vstupných hesiel za účelom rozšírenia prehľadaného priestoru reťazcov. Bežne sa to dosahuje definovaním zoznamu pravidiel popisujúcich transformáciu slova. Tieto pravidlá budú následne aplikované na jednotlivé vstupné slová a tým vzniknú potenciálne nové reťazce, ktoré sa nenachádzajú vo vstupnom slovníku. Tieto pravidlá môžu transformovať slovo rôznymi spôsobmi od pridania prefixu či sufixu cez zmenu veľkostí písmen alebo vynechanie spoluhlások. Mnohé programy zaoberajúce sa útokmi hrubou silou podporujú vlastný jednoduchý jazyk na popísanie týchto pravidiel.

\section{Hybridný útok}
\paragraph{}
V tejto práci sa venujeme implementácii útoku, ktorý je spojením vyššie uvedených. Našim hlavným cieľom je nájdenie správneho hesla k partícii zašifrovanej programom TrueCrypt. Keďže chceme toto heslo nájsť v ľubovoľne veľkom konečnom čase, budeme náš algoritmus implementovať tak, aby vygeneroval všetky možné reťazce zo vstupnej abecedy kratšie ako nami stanovená dĺžka, podobne ako pri úplnom prehľadávaní. Avšak v našom prípade predpokladáme, že na vstupe dostaneme ešte slovník obsahujúci zoznam hesiel. Predpokladáme, že tento zoznam je usporiadaný podľa pravdepodobnosti správnosti hesiel v ňom. Náš algoritmus si na základe týchto hesiel upraví pravdepodobnosti vygenerovania jednotlivých reťazcov aby následne mohol na výstup dávať heslá od najpravdepodobnejšieho po najmenej pravdepodobné. 

\section{SAT Solver}
\paragraph{}
Táto metóda útoku hrubou silou je založená na probléme splniteľnosti boolovského výrazu. Ako vstup tohto algoritmu je boolovský výraz, väčšinou v konjuktívnom normálnom tvare, pre ktorý sa daný algoritmus snaží nájsť také ohodnotenie boolovských premenných, aby všetky formuly tohto výrazu boli pravdivé. Algoritmus postupne rekurzívne prehľadáva všetky možnosti nastavenia jednotlivých premenných. Po nastavení niektorej premennej skontroluje, či žiaden výskyt tejto premennej nespôsobil konflikt, čiže ohodnotil formulu tak, že sa stala nesplniteľnou. V tomto prípade sa algoritmus vráti do momentu kedy bol výraz nekonfliktný a odtiaľ sa snaží postupovať inou cestou.

\paragraph{}
Náš problém sa dá pretransformovať na vstup pre takýto SAT solver. V našom prípade by sme vedeli šifrovací algoritmus prepísať do konjuktívnej normálnej formy. Ako výstup tohto algoritmu je reťazec takmer náhodných bitov, ktoré dopredu poznáme, pretože tieto sú fyzický uložené na disku. Neznámymi v tomto boolovskom výraze sú vstupné dáta a kľúč, pomocou ktorého boli tieto dáta zašifrované. Našim predpokladom je, že veľkú časť týchto dát by sme vedeli určiť, keďže ide o dopredu známe informácie ako reťazec TRUE, použitú verziu programu TrueCrypt a podobne. Vďaka týmto informáciám by sa vedel SAT solver skôr rozhodnúť, že niektoré ohodnotenie premenných nie je možné, kvôli konfliktu, ktorý by vznikol.

\paragraph{}
V našej práci sme túto metódu hlbšie neanalyzovali, keďže existuje viacero prác, ktoré do podrobna rozoberajú správanie týchto solverov v prípade, že majú dopredu určené niektoré bity vstupu alebo výstupu. Taktiež existujú práce, ktoré sa venujú optimalizácii behu SAT solveru, čo zahŕňa optimalizáciu poradia ohodnotenia premenných alebo miesta, na ktoré sa algoritmus vráti v prípade konfliktu. 